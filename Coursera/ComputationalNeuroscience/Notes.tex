%%% LaTeX Template: Article/Thesis/etc. with colored headings and special fonts
%%%
%%% Source: http://www.howtotex.com/
%%% Feel free to distribute this template, but please keep to referal to http://www.howtotex.com/ here.
%%% February 2011

%%%%% Preamble
\documentclass[10pt,a4paper]{article}

\usepackage[T1]{fontenc}
\usepackage[bitstream-charter]{mathdesign}

\usepackage[utf8]{inputenc}							% Input encoding
\usepackage{amsmath}									% Math

\usepackage{xcolor}
\definecolor{bl}{rgb}{0.0,0.2,0.6} 

\definecolor{background-image}{rgb}{1.0,0.9,0.7} 

% Translating the section, part, ... names into French
\usepackage[french]{babel}

% Setup the margins
\usepackage{geometry}
\newgeometry{margin=2cm}

% Customize the hyperlinks
\usepackage{hyperref}
\hypersetup{
    bookmarks=true,         % show bookmarks bar?
    unicode=false,          % non-Latin characters in Acrobat’s bookmarks
    pdftoolbar=true,        % show Acrobat’s toolbar?
    pdfmenubar=true,        % show Acrobat’s menu?
    pdffitwindow=false,     % window fit to page when opened
    pdfstartview={FitH},    % fits the width of the page to the window
    pdftitle={My title},    % title
    pdfauthor={Author},     % author
    pdfsubject={Subject},   % subject of the document
    pdfcreator={Creator},   % creator of the document
    pdfproducer={Producer}, % producer of the document
    pdfkeywords={keyword1} {key2} {key3}, % list of keywords
    pdfnewwindow=true,      % links in new window
    colorlinks=true,       % false: boxed links; true: colored links
    linkcolor=bl,          % color of internal links (change box color with linkbordercolor)
    citecolor=green,        % color of links to bibliography
    filecolor=magenta,      % color of file links
    urlcolor=bl           % color of external links
}

% For getting the caption on the side of an image
\usepackage{calc}
\usepackage{graphicx}
\usepackage{floatrow}
\floatsetup{style=ruled,footposition=caption}

\newcommand{\myfig}[2]{
\begin{figure}[htbp]
\floatbox[{\capbeside\thisfloatsetup{capbesideposition={right,top},capbesidewidth=4cm}}]{figure}[\FBwidth]
         {\caption{#2}\label{fig:#1}}
         {\includegraphics[width=5cm]{#1}}
\end{figure}
}


%\usepackage{sidecap}

\usepackage{sectsty}
\usepackage[compact]{titlesec} 
\allsectionsfont{\color{bl}\scshape\selectfont}

%%%%% Definitions
% Define a new command that prints the title only
\makeatletter							% Begin definition
\def\printtitle{%						% Define command: \printtitle
    {\color{bl} \centering \huge \sc \textbf{\@title}\par}}		% Typesetting
\makeatother							% End definition

\title{Computational Neuroscience \\ 
		\large \vspace*{-10pt} Notes on A. Fairhall / R.P.N. Rao lectures\vspace*{10pt}}

% Define a new command that prints the author(s) only
\makeatletter							% Begin definition
\def\printauthor{%					% Define command: \printauthor
    {\centering \small \@author}}				% Typesetting
\makeatother							% End definition

\author{%
	Jérémy Fix \\
	Jeremy.Fix@Supelec.fr \\
	\vspace{20pt}
	}

% Custom headers and footers
\usepackage{fancyhdr}
	\pagestyle{fancy}					% Enabling the custom headers/footers
\usepackage{lastpage}	
	% Header (empty)
	\lhead{}
	\chead{}
	\rhead{}
	% Footer (you may change this to your own needs)
	\lfoot{\footnotesize }
	\cfoot{}
	\rfoot{\footnotesize page \thepage\ / \pageref{LastPage}}	% "Page 1 of 2"
	\renewcommand{\headrulewidth}{0.0pt}
	\renewcommand{\footrulewidth}{0.4pt}

% Change the abstract environment
\usepackage[runin]{abstract}			% runin option for a run-in title
\setlength\absleftindent{30pt}		% left margin
\setlength\absrightindent{30pt}		% right margin
\abslabeldelim{\quad}						% 
\setlength{\abstitleskip}{-10pt}
\renewcommand{\abstractname}{}
\renewcommand{\abstracttextfont}{\color{bl} \small \slshape}	% slanted text


% Pour gérer les lettrines
\usepackage{lettrine}

%%%%%%%%%%%%%%%%%%%%%%%%%%%%%%%%%%%%%%%%%%%%%%%%%%%%%%%%%%%%%%%%%%%%%%%%%%
%%% Start of the document
%%%%%%%%%%%%%%%%%%%%%%%%%%%%%%%%%%%%%%%%%%%%%%%%%%%%%%%%%%%%%%%%%%%%%%%%%%
\begin{document}
%%% Top of the page: Author, Title and Abstact
\printtitle 

\printauthor

\begin{abstract}
Notes on the Computational Neuroscience lecture of A. Fairhall / R.P.N. Rao
\end{abstract}

\tableofcontents

\pagebreak

\section{Introduction \& Basic Neurobiology}

\section{What do Neurons Encode ? Neural Encoding Models}

\section{Extracting information from Neurons: Neural decoding}

Neural decoding refers to infer the stimulus that causes the neural responses we observe.

Given a feature of a stimulus to identify, we can represent the probability that the stimulus belongs to one category along this feature axis. An illustration of the probability of having a stimulus given some evidence is given by the experiment of Britten et al. 1992. In this experiment, Monkey had to decide in which direction random dot patterns were moving. The difficulty of the task varies and is defined by the coherence of the pattern. If the pattern is moving left ward, the monkey receives a reward if it makes a leftward saccade and conversely with rightward motion. If one looks at the distribution of the firing rates in two cases, when the monkey made a saccade toward the preferred direction and when the monkey made a saccade toward the null direction .....


We introduce $P(r / +)$ the probability of having response $r$ given a stimulus is moving in the preferred direction of the neuron and $P(r / -)$ the probability of having response $r$ given a stimulus is moving in the null direction. Decision takes place when the response is larger than a threshold $z$ to be defined later. We call \emph{false alarms} $P( r \geq z | -)$ i.e. the probability that the neurons fire strongly while the stimulus was moving in the null direction. We call \emph{good calls} $P(r \geq z | +)$ i.e. the probability that the neuron fires strongly while the stimulus is moving in the preferred direction (a good guess). We set the threshold $z$ in order to maximize the probability to be correct, i.e. $P(+) P(r \geq z | +) + P(-) ( 1 - P( r \geq z | -))$ i.e. the probability to fire above threshold when the stimulus is moving in the preferred direction plus the probability to fire below threshold when the stimulus is moving in the null direction. The right choice of the threshold is such that $P(r = z | -) = P(r = z | +)$. The conditional probabilities $P(r / -)$, $P(r / +)$ are called the likelihood. \\

The likelihood ratio is the most efficient statistic, in that it has the most power for a given size (Neyman-Pearson lemna). The likelihood ratio being defined as $p(r/+)/p(+/-)$ and $z$ defined such that $p(r/+)/p(+/-) > 1$.\\

If we decode the neuronal response and compare it to the behavioral response, they match. Therefore, if there is a so close correspondence between neuronal response (at a single cell level?) and behavior, why do we need so many neurons ? (to build the response????)\\

We can consider the likelihood ratio over time. Suppose we have two hypothesis for a stimulus: it might be a tiger or the wind in a bush (breeze). Now take the log of the likelihood ratio : $l(s) = p(s / tiger) / p(s / breeze)$. Given a sample $s$, we can compute the likelihood ratio or its log. If it sounds like the breeze, this makes the likelihood ratio smaller than $1$ and its log negative. As we accumulate samples through time, our guess will fluctuate toward the breeze or the tiger until we reach a threshold indicating we are absolutely confident in our guess. Such a process of accumulated evidence was demonstrated by Kiani(2006). It is similar to Britter(1992) experiment but the monkey now had the time it wants to respond (there is no pressure to answer quickly). Neurons in LIP were recorded during the experiment. Depending on the coherence, we observe different ramps with which the responses are building up (faster when the coherence is larger, slower when the coherence is smaller). When aligned with saccade onset, the responses peak at the small value whatever the coherence.\\

For the tiger/breeze example, we must consider the priors $p(+), p(-)$ which scale the conditionals for taking our decision. An example of including the priors is given by the recordings of Rieke of photoreceptors. The response of photoreceptors is very noisy and, when it receives a photon, is increasing suddenly. One might be interested in the probability of having light given the signal. The priors allow to adjust what is considered as a normal situation for the signal and to decrease false alarms and increase good calls.\\

But priors are not the whole story. We might also include cost. The probability of deciding an action might also depend on how costly it is.\footnote{J'ai un peu le sentiment qu'on mélange beaucoup de chose : le fait de reconnaitre qu'une action devrait être engagée et le coût des actions engageables...}. We incorporate the cost by considering penalties associated with wrong choices. Two losses are therefore considered $\mbox{Loss}_- = \mbox{L}_- P(+/r)$ i.e. the cost of choosing minus while it was a plus and $\mbox{Loss}_+ = \mbox{L}_+ P(-/r)$ i.e. the cost of choosing plus while it was minus. We now act when we can earn more that what we can loose.  This leads to a new criterion for a likelihood ratio test.





\section{Information Theory and Neural Coding}

\end{document}
